%% Plot settings:
% =========================

\usepackage{tikz, pgfplots}

\usetikzlibrary{backgrounds, calc,fit, positioning, patterns, decorations.pathmorphing, decorations.markings, shapes, shapes.geometric, shapes.callouts, arrows.meta, shadings, trees, chains, 3d, shadows}

\tikzset{>={Latex[width=2mm,length=2mm]}, font=\footnotesize}
\tikzset{math3d/.style= {x={(1cm, 0cm)}, y={(.353cm,.353cm)}, z={(0cm,1cm)}}}

\pgfplotsset{compat=newest,
	%tick label style = {font=\sansmath\sffamily},
	%every axis label = {font=\sansmath\sffamily},
	%legend style = {font=\sansmath\sffamily},
	%label style = {font=\sansmath\sffamily},
	grid=major,
	every axis plot/.append style={thick},
	%every axis label/.append style={font=\footnotesize},
	every axis/.append style={font=\footnotesize}, 
	%every node/.append style={font=\footnotesize}, 
}

\newcommand{\inputtikz}[1]{%
	\ifexportfigures
	\tikzexternalenable
	\tikzsetnextfilename{#1}%
	\fi
	\input{figures/#1.tex}%
	\ifexportfigures
	\tikzexternaldisable
	\fi
}

\newcommand{\AxisRotator}[1][rotate=0]{% https://tex.stackexchange.com/a/36607
	\tikz [x=0.25cm,y=0.60cm,line width=.2ex,-stealth,#1] \draw (0,0) arc (-150:150:1 and 1);%
}

\pgfdeclarelayer{background}
\pgfdeclarelayer{foreground}
\pgfsetlayers{background,main,foreground}

\newlength\figureheight
\newlength\figurewidth
\setlength\figureheight{2cm}
\setlength\figurewidth{0.9\textwidth}

\newlength\halffigurewidth 
\setlength\halffigurewidth{0.4\textwidth}

%\newcommand{\setfullfigwidth}{\setlength\figurewidth{.8\textwidth}}
%\newcommand{\sethalfigwidth}{\setlength\figurewidth{.35\textwidth}}
%\newcommand{\setthirdfigwidth}{\setlength\figurewidth{.26\textwidth}}
%\newcommand{\settwothirdsfigwidth}{\setlength\figurewidth{.5\textwidth}}

% Fancy node streching from http://tex.stackexchange.com/a/124507
% Examples:
%
%\node[right=of |(A)(B)(D)]          {Z};
%
%\node[left=of |(A)(B), ultra thick] {Y};
%\node[left=of |(C)(D), minimum width=50pt]          (X) {X};
%
%\node[span vertical=(A)(B), above=of -X] {y};
%
%\node[below=of -(A)(B)(C)(D)] {0};
\makeatletter
\def\pgfutil@firstofmany#1#2\pgf@stop{#1}
\def\pgfutil@secondofmany#1#2\pgf@stop{#2}
\def\tikz@lib@place@of@#1#2#3{%
	\def\pgf@tempa{fit bounding box}%
	\edef\pgf@temp{\expandafter\pgfutil@firstofmany#2\pgf@stop}
	\if\pgf@temp(%
	\tikz@lib@place@fit@scan{#2}{0}%
	\else\if\pgf@temp|
	\expandafter\tikz@lib@place@fit@scan\expandafter{\pgfutil@secondofmany#2\pgf@stop}{1}%
	\else\ifx\pgf@temp\tikz@activebar
	\expandafter\tikz@lib@place@fit@scan\expandafter{\pgfutil@secondofmany#2\pgf@stop}{1}%
	\else\if\pgf@temp-
	\expandafter\tikz@lib@place@fit@scan\expandafter{\pgfutil@secondofmany#2\pgf@stop}{2}%
	\else\if\pgf@temp+
	\expandafter\tikz@lib@place@fit@scan\expandafter{\pgfutil@secondofmany#2\pgf@stop}{3}%
	\else
	\def\pgf@tempa{#2}%
	\fi
	\fi
	\fi
	\fi
	\fi
	\expandafter\tikz@scan@one@point\expandafter\tikz@lib@place@remember\expandafter(\pgf@tempa)%
	\iftikz@shapeborder%
	% Ok, this is relative to a border.
	\iftikz@lib@ignore@size%
	\edef\tikz@node@at{\noexpand\pgfpointanchor{\tikz@shapeborder@name}{center}}%
	\def\tikz@anchor{center}%
	\else%
	\edef\tikz@node@at{\noexpand\pgfpointanchor{\tikz@shapeborder@name}{#3}}%
	\fi%
	\fi%
	\edef\tikz@lib@place@nums{#1}%
}
\def\tikz@lib@place@fit@scan#1#2{
	\pgf@xb=-16000pt\relax%
	\pgf@xa=16000pt\relax%
	\pgf@yb=-16000pt\relax%
	\pgf@ya=16000pt\relax%
	\if\pgfutil@firstofmany#1\pgf@stop(%
	\tikz@lib@fit@scan#1\pgf@stop%
	\else
	\tikz@lib@fit@scan(#1)\pgf@stop
	\fi
	\ifdim\pgf@xa>\pgf@xa
	% shouldn't happen
	\else
	\expandafter\def\csname pgf@sh@ns@fit bounding box\endcsname{rectangle}%
	\expandafter\edef\csname pgf@sh@np@fit bounding box\endcsname{%
		\def\noexpand\southwest{\noexpand\pgfqpoint{\the\pgf@xa}{\the\pgf@ya}}%
		\def\noexpand\northeast{\noexpand\pgfqpoint{\the\pgf@xb}{\the\pgf@yb}}%
	}%
	\expandafter\def\csname pgf@sh@nt@fit bounding box\endcsname{{1}{0}{0}{1}{0pt}{0pt}}%
	\expandafter\def\csname pgf@sh@pi@fit bounding box\endcsname{\pgfpictureid}%
	\ifcase#2\relax
	\or % 1 = vertical
	\pgf@y=\pgf@yb%
	\advance\pgf@y by-\pgf@ya%
	\edef\pgf@marshal{\noexpand\tikzset{minimum height={\the\pgf@y-2*(\noexpand\pgfkeysvalueof{/pgf/outer ysep})}}}%
	\pgf@marshal
	\or % 2 = horizontal
	\pgf@x=\pgf@xb%
	\advance\pgf@x by-\pgf@xa%
	\edef\pgf@marshal{\noexpand\tikzset{minimum width={\the\pgf@x-2*(\noexpand\pgfkeysvalueof{/pgf/outer xsep})}}}%
	\pgf@marshal
	\or % 3 = both directions
	\pgf@y=\pgf@yb%
	\advance\pgf@y by-\pgf@ya%
	\pgf@x=\pgf@xb%
	\advance\pgf@x by-\pgf@xa%
	\edef\pgf@marshal{\noexpand\tikzset{minimum height={\the\pgf@y-2*(\noexpand\pgfkeysvalueof{/pgf/outer ysep})},minimum width={\the\pgf@x-2*(\noexpand\pgfkeysvalueof{/pgf/outer xsep})}}}%
	\pgf@marshal
	\fi
	\fi
}
\tikzset{
	fit bounding box/.code={\tikz@lib@place@fit@scan{#1}{0}},
	span vertical/.code={\tikz@lib@place@fit@scan{#1}{1}},
	span horizontal/.code={\tikz@lib@place@fit@scan{#1}{2}},
	span/.code={\tikz@lib@place@fit@scan{#1}{3}}}
